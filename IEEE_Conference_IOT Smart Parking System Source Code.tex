\documentclass[conference]{IEEEtran}
\IEEEoverridecommandlockouts
% The preceding line is only needed to identify funding in the first footnote. If that is unneeded, please comment it out.
\usepackage{cite}
\usepackage{amsmath,amssymb,amsfonts}
\usepackage{algorithmic}
\usepackage{graphicx}
\usepackage{textcomp}
\usepackage{xcolor}
\def\BibTeX{{\rm B\kern-.05em{\sc i\kern-.025em b}\kern-.08em
    T\kern-.1667em\lower.7ex\hbox{E}\kern-.125emX}}
\begin{document}

\title{IOT Smart parking system*\\
{\footnotesize \textsuperscript{*}Note: Sub-titles are not captured in Xplore and
should not be used}
\thanks{Identify applicable funding agency here. If none, delete this.}
}

\author{\IEEEauthorblockN{1\textsuperscript{st} Shahbaj Rahman}
\IEEEauthorblockA{\textit{dept. of CSE} \\
\textit{European University Of Bangladesh}\\
Gabtoli,Dhaka,Bangladesh \\
ID : 200122087}
\and
\IEEEauthorblockN{2\textsuperscript{nd} Esrat Jahan}
\IEEEauthorblockA{\textit{dept. of CSE} \\
\textit{European University Of Bangladesh}\\
Gabtoli,Dhaka,Bangladesh \\
ID : 200122144}
\and
\IEEEauthorblockN{3\textsuperscript{rd}  A.K.M Saharuzzaman}
\IEEEauthorblockA{\textit{dept. of CSE} \\
\textit{European University Of Bangladesh}\\
Gabtoli,Dhaka,Bangladesh \\
ID : 200122106}
\and
\IEEEauthorblockN{4\textsuperscript{th}  MD:Miraz Mridha}
\IEEEauthorblockA{\textit{dept. of CSE} \\
\textit{European University Of Bangladesh}\\
Gabtoli,Dhaka,Bangladesh \\
ID : 200122146}
\and
\IEEEauthorblockN{5\textsuperscript{th}  MD:Moniruzzaman}
\IEEEauthorblockA{\textit{dept. of CSE} \\
\textit{European University Of Bangladesh}\\
Gabtoli,Dhaka,Bangladesh \\
ID : 200122090}
\and
\IEEEauthorblockN{6\textsuperscript{th}  MD:Shohidul Islam}
\IEEEauthorblockA{\textit{dept. of CSE} \\
\textit{European University Of Bangladesh}\\
Gabtoli,Dhaka,Bangladesh \\
ID : 200122111}
}

\maketitle

\begin{abstract}
This document is a model and instructions for \LaTeX.
This and the IEEEtran.cls file define the components of your paper [title, text, heads, etc.]. *CRITICAL: Do Not Use Symbols, Special Characters, Footnotes, 
or Math in Paper Title or Abstract.
\end{abstract}

\begin{IEEEkeywords}
component, formatting, style, styling, insert
\end{IEEEkeywords}

\section{Introduction}
IOT : The Internet of think or IOT is a system of interrelated computing devices, mechanical and digital machines \LaTeX.
 Parking System : A car parking system is a mechanical device that multiplies parking capacity inside a parking lot. Parking systems are generally powered by electric motors or hydraulic pumps that move vehicles into a storage position
 

\section{What is IOT Smart Parking System?}\label{AA}

\subsection{IOT smart Parking System parking lot problems and solutions}
\begin{itemize}
\item Increase parking supply.
\item Establish minimum parking requirements.
\item Subsidizing off-street parking.
\item Increasing on-street parking AND
Increase on-street/curbside parking provision.

\end{itemize}

\subsection{An IOT based smart parking system}
An IOT based smart parking system, also known as
 a connected parking system, is a centralized
 management system that allows drivers to use
 a smartphone app to search for and reserve
 a parking spot.

\subsection{What is problem?}\label{AA}

\subsection{How to work this IOT smart Parking System ?}
\begin{itemize}
\item Here is 3 concept of this system.
\item Space Detector.
\item Data Transfer.
\item Data Receiver from smart device.

\end{itemize}


\subsection{How to work this IOT smart Parking System ?}\label{SCM}
\begin{itemize}

\item FREE Space.
\item No Space Free.
\item Dirty Space Free.
\end{itemize}

\subsection{SYSTEM ARCHITECTURE}
This section describes the high level architecture for the smart
parking system along with a mathematical model. The parking
system that we propose comprises of various actors that work
in sync with one another. Below is the mathematical model that
defines our smart parking system.
Table 1: N
\subsection{NEED FOR IOT-CLOUD INTEGRATION}
Cloud computing and IoT have witnessed large evolution. Both
the technologies have their advantages, however several mutual
advantages can be foreseen from their integration. On one
hand, IoT can address its technological constraints such as
storage, processing and energy by leveraging the unlimited
capabilities and resources of Cloud[4]. On the other hand,
Cloud can also extend its reach to deal with real world entities
in a more distributed and dynamic fashion by the use of IoT.
Basically, the Cloud acts as an intermediate between things and
applications, in order to hide all the complexities and
functionalities necessary for running the application. Below are
some of the factors that led to the amalgamation of Cloud and
IoT

Mi ĺ X(T,C,P,U,S) // Driver provides input to the input
function
X()ĺF(S,T) // Input function notifies the computation
function
X()ĺI(P,C,U) // Input function notifies the identity function
Oi= F(S,T)ĺY() // Computation function notifies the output
function and the resultant is stored in form of the occupancy
rate.
TCP/IP protocol. It is designed to establish
connections across remote locations where limited
amount of data needs to be transferred or in cases of
low bandwidth availability

\subsection{Storage capacity}
IoT comprises of a large number of information sources (things), which produce huge amounts of non-structured or semi-structured data. As a result IoT requires collecting, accessing, processing, visualizing and sharing large amounts of data[14]. Cloud provides unlimited, low-cost, and on-demand storage capacity, thus making it the best and most cost effective solution to deal with data generated by IoT


\section*{Computation power}

The devices being used under IoT have limited processing capabilities. Data collected from various sensors is usually transmitted to more powerful nodes where its aggregation and processing can be done[18]. The computation needs of IoT can be addressed by the use of unlimited processing capabilities and on-demand model of Cloud. With the help of cloud computing, IoT systems could perform real-time processing of data thus facilitating highly responsive applications

\section*{Communication resources}

The basic functionality of IoT is to make IP-enabled devices communicate with one another through dedicated set of hardware. Cloud computing offers cheap and effective ways of connecting, tracking, and managing devices from anywhere over the internet[16]. By the use of built-in applications IoT systems could monitor and control things on a real-time basis through remote locations
\section*{Scalability}
Cloud provides a scalable approach towards IoT. It allows increase or decrease in resources in a dynamic fashion. Any number of “things” could be added or subtracted from the system when cloud integration is provided[22]. The cloud allocates resources in accordance with the requirements of things and applications
\section*{Parking Sensors}
For our parking system we have made use of sensors like Infrared, Passive Infrared(PIR) and Ultrasonic Sensors. The work of these sensors is the same i.e. to sense the parking area and determine whether a parking slot is vacant or not. In this case we are using ultrasonic sensors to detect the presence of a car. The ultrasonic sensors are wirelessly connected to raspberry pi using the ESP8266 chip. An ESP8266 WiFi chip comprises of a self contained SOC with integrated TCP/IP protocol stack that allows any microcontroller to access a WiFi network. The sensors are connected to a 5V supply either from raspberry pi or an external source. External source being more preferable. 

\section*{Processing Unit}
: It comprises of Raspberry pi which is a processor on chip. The processing unit acts like an intermediate between the sensors and cloud. All the sensors are wirelessly connected to the processing unit. A single raspberry pi unit comprises of 26 GPIO pins i.e. 26 different sensors can be connected to it. However we can increase this number by attaching a multiplexer (MUX) to it. It is essential that the ground of raspberry pi and sensors must be connected in order to transfer data using the GPIO pins. There is a python script running on the chip that checks the status of various GPIO pins and updates this information onto the cloud. Data collected from various sensors is sent to the raspberry pi through the esp8266 chip.
\section*{Flow chart of the system}
We conducted an experiment in order to depict the working of our system at every stage from checking the availability of parking space to actually park a car in a vacant parking slot. This is done by implementing the smart parking system in the parking area of a shopping mall. Below are the steps that a driver needs to follow in order to park its car using our parking system. 

\section*{Flow chart of the system}Step 1: Insall the smart parking application on your mobile device.  
Step 2: With the help of the mobile app search for a parking area on and around your destination.  
Step 3: Select a particular parking area.  
Step 4: Browse through the various parking slots available in that parking area. x 
Step 5: Select a particular parking slot.  
Step 6: Select the amount of time (in hours) for which you would like to park your car for.  
Step 7: Pay the parking charges either with your wallet or your credit card.  
Step 8: Once you have successfully parked your car in the selected parking slot, confirm your occupancy using the mobile application. 

\end{document}
